\section{Target System}
%%%%%%%%%%%%%%%%%%%%%%%%%%%%%%%%%%%%%%%%%%%%%%%%%%%%%%%%%%%%%%%%%%%%
\begin{frame}
  \frametitle{Target System: \texttt{simcl1} - Hardware}
  \texttt{simcl1} is a small cluster consisting of one login node and $\num{4}$ compute nodes. \\
  \textbf{Attention:} \textbf{do not} perform \textbf{any} calculations on the login node!
  \vfill
  Available hardware per compute node:
  \begin{itemize}
      \item dual-socket \link{https://www.amd.com/en/products/cpu/amd-epyc-9274f}{AMD EPYC 9274F 24-Core @ 4.05GHz} CPUs
      \begin{itemize}
          \item $\num{48}$ physical cores
          \item $\num{96}$ hyper threads
          \item $\num{2}$ sockets
      \end{itemize}
      \item $\SI{384}{\gibi\byte}$ DDR5 RAM
      \item $\SI{1.5}{\tera\byte}$ \texttt{/data/scratch} space (\textbf{not shared} across nodes of the cluster)\\
      \itFollows{If you want to use the \texttt{/scratch} space instead of your \texttt{\$HOME} directory, you must create the \texttt{/data/scratch/FaPra/\$\{GROUP\_NAME\}} custom directory and delete it afterwards!}
      \item inter-node connection via \SI{10}{\giga\byte} Ethernet
      \item nodes 1 and 2 contain an NVIDIA A30 GPU each
      \item node 3 contains an AMD Mi210 GPU
  \end{itemize}
\end{frame}
%%%%%%%%%%%%%%%%%%%%%%%%%%%%%%%%%%%%%%%%%%%%%%%%%%%%%%%%%%%%%%%%%%%%

\subsection{Connecting to simcl1}
%%%%%%%%%%%%%%%%%%%%%%%%%%%%%%%%%%%%%%%%%%%%%%%%%%%%%%%%%%%%%%%%%%%%
\begin{frame}[fragile]{Target System: \texttt{simcl1} - Connecting to the Cluster - 1}
    First login to our server:
    \begin{enumerate}
        \item \mintinline{sh}{ssh ${USERNAME}@ipvslogin.informatik.uni-stuttgart.de}
        \item You will be prompted to input your password.
        \item Change your initial password: See \mintinline{sh}{passwd} and follow the displayed instructions. This \textbf{must} be done during your first login.
        \item Disconnect from the server by either typing \mintinline{sh}{exit} or using \texttt{Strg + d}.
    \end{enumerate}
    \pause
    \vfill
    Connecting to the \texttt{simcl1} cluster's login node using an ssh key:
    \begin{enumerate}
        \item \link{https://docs.oracle.com/en/cloud/cloud-at-customer/occ-get-started/generate-ssh-key-pair.html}{Generate a new ssh key pair} (if none already exists).
        \item Copy your public ssh key to our login server via:
        {
            \setfontsize{9pt}
            \mintinline{sh}{ssh-copy-id -i ~/.ssh/id_rsa ${USERNAME}@ipvslogin.informatik.uni-stuttgart.de}
        }
        \textbf{Note:} you will be prompted to enter your passwords.
    \end{enumerate}
\end{frame}
%%%%%%%%%%%%%%%%%%%%%%%%%%%%%%%%%%%%%%%%%%%%%%%%%%%%%%%%%%%%%%%%%%%%

%%%%%%%%%%%%%%%%%%%%%%%%%%%%%%%%%%%%%%%%%%%%%%%%%%%%%%%%%%%%%%%%%%%%
\begin{frame}[fragile]{Target System: \texttt{simcl1} - Connecting to the Cluster - 2}
    \begin{enumerate}
        \setcounter{enumi}{2}
        \item Add the following lines to your \texttt{\~{}/.ssh/config} file (on your local machine):\vspace*{-.5em}
        \begin{minted}{bash}
host ipvslogin
    ForwardAgent yes
    hostname ipvslogin.informatik.uni-stuttgart.de
    user ${USERNAME}
    IdentityFile ~/.ssh/id_rsa

host simcl1
    ForwardAgent yes
    ProxyJump ipvslogin
    hostname
    user ${USERNAME}
    IdentityFile ~/.ssh/id_rsa
        \end{minted}
        \item Now you can login to \texttt{simcl1} by simply using: \mintinline{sh}{ssh simcl1}
    \end{enumerate}
\end{frame}
%%%%%%%%%%%%%%%%%%%%%%%%%%%%%%%%%%%%%%%%%%%%%%%%%%%%%%%%%%%%%%%%%%%%
