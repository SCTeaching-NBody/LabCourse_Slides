\section{Visualization via ParaView}
%%%%%%%%%%%%%%%%%%%%%%%%%%%%%%%%%%%%%%%%%%%%%%%%%%%%%%%%%%%%%%%%%%%%
\begin{frame}[fragile]
  \frametitle{Visualization via ParaView}
  \begin{itemize}
      \item Visualizing the data via \link{https://www.paraview.org/}{ParaView}.
      \item Create one \texttt{.vtp} file, containing information about the current simulation state (e.g., body mass, position, energies of the system, $\dots$) for every visualization step on \textbf{every} MPI rank containing only the data it is responsible for.
      \item Create one \texttt{.pvd} file for the whole simulation on \textbf{one} MPI rank, referencing all \texttt{.vtp} files including the respective time step.
  \end{itemize}
\end{frame}
%%%%%%%%%%%%%%%%%%%%%%%%%%%%%%%%%%%%%%%%%%%%%%%%%%%%%%%%%%%%%%%%%%%%

\subsection{.pvd and .vtp Files}
%%%%%%%%%%%%%%%%%%%%%%%%%%%%%%%%%%%%%%%%%%%%%%%%%%%%%%%%%%%%%%%%%%%%
\begin{frame}[fragile]
  \frametitle{Example of a \texttt{.vtp} File with Two Bodies - 1}
  \setfontsize{5.3pt}
    \begin{minted}[bgcolor=lightgray!30,obeytabs=true,tabsize=2]{xml}
<?xml version="1.0"?>
<VTKFile type="PolyData" version="0.1" byte_order="LittleEndian" header_type="UInt64">
	<PolyData>
		<Piece NumberOfPoints="2" NumberOfVerts="2">
			<Points>
				<DataArray type="Float64" Name="position" NumberOfComponents="3" format="ascii">
                    0 0 0
                    -0.140728 -0.443901 -0.0233456
				</DataArray>
			</Points>
			<PointData>
        <DataArray type="Int32" Name="body_id" NumberOfComponents="1" format="ascii">
                    0
                    1
        </DataArray>
				<DataArray type="Float64" Name="velocity" NumberOfComponents="3" format="ascii">
                    5.37193e-06 -7.4078e-06 -9.4222e-08
                    0.0211745 -0.00710548 -0.00252296
				</DataArray>
				<DataArray type="Float64" Name="acceleration" NumberOfComponents="1" format="ascii">
                    9.15107e-06
                    0.022477
				</DataArray>
				<DataArray type="Float64" Name="mass" NumberOfComponents="1" format="ascii">
                    1.98847e+30
                    3.30101e+23
				</DataArray>
				<DataArray type="String" Name="name" NumberOfComponents="1" format="ascii">
                    83 117 110 0
                    77 101 114 99 117 114 121 0
				</DataArray>
  \end{minted}
\end{frame}
%%%%%%%%%%%%%%%%%%%%%%%%%%%%%%%%%%%%%%%%%%%%%%%%%%%%%%%%%%%%%%%%%%%%

%%%%%%%%%%%%%%%%%%%%%%%%%%%%%%%%%%%%%%%%%%%%%%%%%%%%%%%%%%%%%%%%%%%%
\begin{frame}[fragile]
  \frametitle{Example of a \texttt{.vtp} File with Two Bodies - 2}
  \setfontsize{5.3pt}
    \begin{minted}[bgcolor=lightgray!30,obeytabs=true,tabsize=2]{xml}
				<DataArray type="Int32" Name="orbit_class" NumberOfComponents="1" format="ascii">
                    0
                    1
				</DataArray>
			</PointData>
			<Verts>
				<DataArray type="Int64" Name="offsets">
                    1 2 
				</DataArray>
				<DataArray type="Int64" Name="connectivity">
                    0 1 
				</DataArray>
			</Verts>
		</Piece>
		<FieldData>
			<DataArray type="Float64" Name="kinetic energy" NumberOfTuples="1" format="ascii">
                    7.18489e+22
			</DataArray>
			<DataArray type="Float64" Name="potential energy" NumberOfTuples="1" format="ascii">
                    -1.37988e+23
			</DataArray>
			<DataArray type="Float64" Name="total energy" NumberOfTuples="1" format="ascii">
                    -6.6139e+22
			</DataArray>
			<DataArray type="Float64" Name="virial equilibrium" NumberOfTuples="1" format="ascii">
                    1.04138
			</DataArray>
		</FieldData>
	</PolyData>
</VTKFile>
  \end{minted}
\end{frame}
%%%%%%%%%%%%%%%%%%%%%%%%%%%%%%%%%%%%%%%%%%%%%%%%%%%%%%%%%%%%%%%%%%%%

%%%%%%%%%%%%%%%%%%%%%%%%%%%%%%%%%%%%%%%%%%%%%%%%%%%%%%%%%%%%%%%%%%%%
\begin{frame}
    \frametitle{\texttt{.vtp} File General Remarks}
    \begin{itemize}
        \item The \texttt{body\_id} must be unique for each body \textbf{across all} MPI ranks.
        \item The \texttt{name} must be saved converting each character to its ASCII representation and adding a \texttt{NULL} terminator. \\[.25em]
        Example:\quad
        \begin{tabular}{|c|c|c|c|}
            \hline
            S & u & n & \\\hline
            83 & 117 & 110 & 0 \\\hline 
        \end{tabular}
        \item The orbit~class is an integer corresponding to one of the classes on the bonus material
        \hyperlink{orbit_classes}{slide \ref{orbit_classes}}.
        \itFollows{Devise a suitable mapping!}
        \item The \texttt{offsets} and \texttt{connectivity} must start at \texttt{1} and \texttt{0} respectively on \textbf{each} MPI rank.
    \end{itemize}
\end{frame}
%%%%%%%%%%%%%%%%%%%%%%%%%%%%%%%%%%%%%%%%%%%%%%%%%%%%%%%%%%%%%%%%%%%%

%%%%%%%%%%%%%%%%%%%%%%%%%%%%%%%%%%%%%%%%%%%%%%%%%%%%%%%%%%%%%%%%%%%%
\begin{frame}[fragile]
  \frametitle{Example of a \texttt{.pvd} File}
  \textbf{Note:} \mintinline{xml}{timestep} \textbf{does not} correspond to the current time step number but to the current simulation time (in earth days)!
  \vfill
  Example with a visualization step width $vs = \SI{5}{\day}$ using two MPI ranks:
  \setfontsize{7.2pt}
  \begin{minted}[bgcolor=lightgray!30,obeytabs=true,tabsize=2]{xml}
<?xml version="1.0"?>
<VTKFile type="Collection" version="0.1" byte_order="LittleEndian" compressor="vtkZLibDataCompressor">
	<Collection>
		<DataSet timestep="0.0" group="" part="0" file="time_series/0/sim.0.vtp"/>
		<DataSet timestep="0.0" group="" part="1" file="time_series/1/sim.0.vtp"/>
		<DataSet timestep="5.0" group="" part="0" file="time_series/0/sim.1.vtp"/>
		<DataSet timestep="5.0" group="" part="1" file="time_series/1/sim.1.vtp"/>
		<DataSet timestep="10.0" group="" part="0" file="time_series/0/sim.2.vtp"/>
		<DataSet timestep="10.0" group="" part="1" file="time_series/1/sim.2.vtp"/>
		<DataSet timestep="15.0" group="" part="0" file="time_series/0/sim.3.vtp"/>
		<DataSet timestep="15.0" group="" part="1" file="time_series/1/sim.3.vtp"/>
	</Collection>
</VTKFile>
  \end{minted}
\end{frame}
%%%%%%%%%%%%%%%%%%%%%%%%%%%%%%%%%%%%%%%%%%%%%%%%%%%%%%%%%%%%%%%%%%%%

\subsection{ParaView}
%%%%%%%%%%%%%%%%%%%%%%%%%%%%%%%%%%%%%%%%%%%%%%%%%%%%%%%%%%%%%%%%%%%%
\begin{frame}[fragile]
	\frametitle{ParaView Basics - 1}
	Most important ParaView views and options:
	\setlength{\leftmargini}{-0.4cm}
    \setlength{\leftmarginii}{-0.4cm}
    \begin{description}
      \item[View $\rightarrow$ Properties] used for general visualization properties
      \begin{itemize}
          \item Use \texttt{Points} as \textit{Representation}.
          \item Use an appropriate \textit{Point Size}.
      \end{itemize}
      \item[View $\rightarrow$ Find Data] used to extract data satisfying a specific property
      \begin{itemize}
          \item Can be used to create different groups for, e.g., different orbit classes (these groups can then be visualized with different colors or \textit{Point Sizes}!).
          \item Use \texttt{name} in the \textit{Point Labels} drop-down menu to print the body name during the simulation.
          \item Can also be used to search for a specific body (\textbf{Note:} unfortunately it is not possible to search by the body name in ParaView!).
      \end{itemize}
      \item[View $\rightarrow$ Animation View] used for more control over the simulation visualization
      \begin{itemize}
          \item Select a previously generated (via \enquote{Find Data}) extraction.
          \item Select \texttt{Camera} and \texttt{Follow Data} in the two drop-downs and click the \textit{+}.
          \item ParaView now follows the specified data during the simulation!
      \end{itemize}
    \end{description}
\end{frame}
%%%%%%%%%%%%%%%%%%%%%%%%%%%%%%%%%%%%%%%%%%%%%%%%%%%%%%%%%%%%%%%%%%%%

%%%%%%%%%%%%%%%%%%%%%%%%%%%%%%%%%%%%%%%%%%%%%%%%%%%%%%%%%%%%%%%%%%%%
\begin{frame}[fragile]
	\frametitle{ParaView Basics - 2}
	\setlength{\leftmargini}{-0.4cm}
    \setlength{\leftmarginii}{-0.4cm}
    \begin{description}
      \item[Automate ParaView Tasks] \hfill\\
        \begin{itemize}
            \item Start recording via \textit{Tools $\rightarrow$ Start Trace}.
            \item Do all the stuff you want to be automated.
            \item Stop recording via \textit{Tools $\rightarrow$ Stop Trace}.
            \item A new window opens where you can further customize the macro.
            \item Save the script as macro via \textit{File $\rightarrow$ Save As Macro$\dots$}.
            \item The new macro can now be found under \textit{Macros}.
        \end{itemize}
      \item[Filters $\rightarrow$ Data Analysis $\rightarrow$ Plot Global Variables Over Time] \hfill\\\hspace{-.5cm}
        used to display the energy plot
      \item[File $\rightarrow$ Save Screenshot...] \hfill\\\hspace{-.5cm}
        used to save the \textbf{current} time step as an image
      \item[File $\rightarrow$ Save Animation...] \hfill\\\hspace{-.5cm}
        used the save the \textbf{whole} simulation as a video
    \end{description}
\end{frame}
%%%%%%%%%%%%%%%%%%%%%%%%%%%%%%%%%%%%%%%%%%%%%%%%%%%%%%%%%%%%%%%%%%%%