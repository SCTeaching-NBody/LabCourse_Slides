\section{Simulation}
%%%%%%%%%%%%%%%%%%%%%%%%%%%%%%%%%%%%%%%%%%%%%%%%%%%%%%%%%%%%%%%%%%%%
\begin{frame}[fragile]
  \frametitle{Running the Simulation}

The simulation must be executable on the command line as follows, whereby the command line parameters may be given in an arbitrary order!
\begin{center}
    \setfontsize{9pt}
    \mintinline{text}{./simulate --file data.csv --dt 1h --t_end 1y --vs 1d --vs_dir sim_out --theta 1.05}
    \mintinline{text}{./simulate  --dt 1h --vs 1d --t_end 1y --theta 1.05 --file data.csv --vs_dir sim_out}
\end{center}

\begin{description}
  \item[\texttt{--file}] the used simulation data file
  \item[\texttt{--dt}] the time step width $\Delta t$
  \item[\texttt{--t\_end}] the end time of the simulation
  \item[\texttt{--vs}] the visualization step width
  \item[\texttt{--vs\_dir}] the top most output directory for all visualization files
  \item[\texttt{--theta}] the threshold in the Barnes-Hut algorithm
\end{description}
\vfill
The above invocation simulates \mintinline{text}{data.csv} for one earth year with a resolution of one hour visualizing each day and writing all visualization files to the folder \mintinline{text}{sim_out/}.
\end{frame}
%%%%%%%%%%%%%%%%%%%%%%%%%%%%%%%%%%%%%%%%%%%%%%%%%%%%%%%%%%%%%%%%%%%%

%%%%%%%%%%%%%%%%%%%%%%%%%%%%%%%%%%%%%%%%%%%%%%%%%%%%%%%%%%%%%%%%%%%%
\begin{frame}[fragile]
  \frametitle{Running the Simulation - Time Duration Parameters}

    For a more intuitive command line experience, the values of the three command line parameters \texttt{--dt}, \texttt{--t\_end}, and \texttt{--vs} are positive, decimal numbers followed by one of \texttt{h} (hours), \texttt{d} (days), \texttt{m} (months), or \texttt{y} (years).
    \vfill
    Examples for valid values are: \SI{1}{\hour}, \SI{12}{\day}, \SI{3.5}{\month}, or \SI{1.25}{\year}. \\[.4em]
    Examples for invalid values are: \SI{-2}{\hour}, \SI{60}{\second}, or \num{1.5}\,\texttt{hd}.
    \vfill
    \pause
    \textbf{Note:} you should convert all values into earth days for your internal calculations:
    \begin{itemize}
        \item $\SI{1}{\hour} \equiv \SI{0.0416667}{\day}$
        \item $\SI{1}{\month} \equiv \SI{30.4167}{\day}$
        \item $\SI{1}{\year} \equiv \SI{365.25}{\day}$
    \end{itemize}
\end{frame}
%%%%%%%%%%%%%%%%%%%%%%%%%%%%%%%%%%%%%%%%%%%%%%%%%%%%%%%%%%%%%%%%%%%%

\subsection{Validation}
%%%%%%%%%%%%%%%%%%%%%%%%%%%%%%%%%%%%%%%%%%%%%%%%%%%%%%%%%%%%%%%%%%%%
\begin{frame}[fragile]
  \frametitle{Validation Hints for the Simulation}
  \begin{itemize}
    \item Use the ParaView visualization to validate your simulation:
    \begin{itemize}
        \item Is the simulation stable, i.e., do all planets and their moons have stable orbits?
        \item Use the known orbital periods of objects:
        \begin{itemize}
            \item Earth approx. \SI{1}{\year} (\SI{365.25}{\day})
            \item Mars approx. \SI{687}{\day}
            \item Jupiter approx. \SI{12}{\year} (\SI{4383}{\day}) 
            \item Europa (moon) approx. \SI{85}{\hour} (\SI{3.54}{\day})
        \end{itemize}
    \end{itemize}
    \item Use the Virial Equilibrium and the laws of mass and energy conservation as additional mechanisms to detect errors.
  \end{itemize}
  \vfill
  \pause
  \textbf{Attention!:}\\
  It may happen that Pluto and its moons will not form a stable system, i.e., some moons (mainly Kerberos and Hydra) will be ejected from the system. This is most likely \textbf{not} a bug in your simulation, but the consequence of insufficient data from JPL's Horizons website.\\[-.5em]
  \begin{spacing}{0.8}
      \setfontsize{8pt}
      For more information see: \\
      \texttt{S. Kenyon, B. Bromley: A Pluto-Charon Sonata IV. Improved Constraints on the Dynamical Behavior and Masses of the Small Satellites\textquotedblright (2022) (\url{https://arxiv.org/pdf/2204.04226.pdf})}
    \end{spacing}
\end{frame}
%%%%%%%%%%%%%%%%%%%%%%%%%%%%%%%%%%%%%%%%%%%%%%%%%%%%%%%%%%%%%%%%%%%%