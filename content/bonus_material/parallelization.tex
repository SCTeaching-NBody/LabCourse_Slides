\section{Parallelization}
%%%%%%%%%%%%%%%%%%%%%%%%%%%%%%%%%%%%%%%%%%%%%%%%%%%%%%%%%%%%%%%%%%%%
\againframe<2>{complete_simulation}
%%%%%%%%%%%%%%%%%%%%%%%%%%%%%%%%%%%%%%%%%%%%%%%%%%%%%%%%%%%%%%%%%%%%


%%%%%%%%%%%%%%%%%%%%%%%%%%%%%%%%%%%%%%%%%%%%%%%%%%%%%%%%%%%%%%%%%%%%
\begin{frame}{Parallelization}

    \begin{flushleft}
        \begin{tikzpicture}[scale=0.9]

% CPUs
\begin{scope}[shift={(0, 0.25)}]
    \draw (0, 0) rectangle (2, 2);
    \node at (1, .25) {CPU 0};
    \draw (0.15, 1.35) rectangle (0.85, 1.85);
    \draw[fill=red] (0.225, 1.475) rectangle (0.45, 1.725);
    
    \visible<2->{
        \draw[fill=BurntOrange] (0.55, 1.475) rectangle (0.775, 1.725);
        
        \draw (1.15, 1.35) rectangle (1.85, 1.85);
        \draw[fill=red] (1.225, 1.475) rectangle (1.45, 1.725);
        \draw[fill=BurntOrange] (1.55, 1.475) rectangle (1.775, 1.725);
        
        \draw (0.15, 0.65) rectangle (0.85, 1.15);
        \draw[fill=red] (0.225, 0.775) rectangle (0.45, 1.025);
        \draw[fill=BurntOrange] (0.55, 0.775) rectangle (0.775, 1.025);
        
        \draw (1.15, 0.65) rectangle (1.85, 1.15);
        \draw[fill=red] (1.225, 0.775) rectangle (1.45, 1.025);
        \draw[fill=BurntOrange] (1.55, 0.775) rectangle (1.775, 1.025);
    };
\end{scope}
% \draw (0, 0.25) pic{cpu={0}};

\visible<3->{
    \draw (3.5, -2.2) pic{cpu={1}};
    \draw (2, -5) pic{cpu={2}};
    \draw (-2, -5) pic{cpu={3}};
    \draw (-3.5, -2.2) pic{cpu={4}};
    
    % Interconnects
    \draw[dashed] (-1.5, -1.1) -- (1.0, -1.625 + 0.25);
    \draw[dashed] (1.0, 0.25) -- (1.0, -1.625 + 0.25);
    \draw[dashed] (3.5, -1.1) -- (1.0, -1.625 + 0.25);
    \draw[dashed] (3, -3) -- (1.0, -1.625 + 0.25);
    \draw[dashed] (-1, -3) -- (1.0, -1.625 + 0.25);
    \draw[fill=white] (1.0, -1.625 + 0.25) circle (.4cm);
}
        \end{tikzpicture}
    \end{flushleft}
    \begin{tikzpicture}[overlay, shift={(10, 6)}]
        \visible<1->{
            \node[anchor=west, align=left, minimum size=2cm] at (0, 0) {Normal:\\\num{1} core};
        };
        \visible<2->{
            \node[anchor=west, align=left, minimum size=2cm] at (0.25, -1.6) {OpenMP:\\\num{4} cores (\num{8} HT)};
        };
        \visible<3->{
            \node<3->[anchor=west, align=left, minimum size=2cm] at (0.25, -3.3) {MPI + OpenMP:\\\num{20} cores (\num{40} HT)\\[-.25em]on \num{5} nodes};
        };
    \end{tikzpicture}
\end{frame}
%%%%%%%%%%%%%%%%%%%%%%%%%%%%%%%%%%%%%%%%%%%%%%%%%%%%%%%%%%%%%%%%%%%%

%%%%%%%%%%%%%%%%%%%%%%%%%%%%%%%%%%%%%%%%%%%%%%%%%%%%%%%%%%%%%%%%%%%%
\begin{frame}[fragile]{Parallelization Hints}
    \begin{itemize}
        \item It is allowed and highly advised \textbf{to hold the data redundant on each node}.
        \item Tree refinements on one level and the ones below are independent of each other.
        \item \textbf{Hint}: start with a sequential implementation, but keep the parallelization in mind (especially when choosing the used data structures).
        \item All data structures that need to be communicated via MPI must be serialized in some way.
        \item Pointers can not easily point into another MPI rank's memory.
        \item Pointer addresses change if you copy a structure.
        \item A structure with relative offsets can be more easily merged together than a structure with absolute offsets.
        \item Can the Barnes-Hut tree construction also be parallelized? 
    \end{itemize}
    \vfill
    \pause
    \DisplayRightArrow Think about \textbf{what} can be computed independently from each other! \\
    \DisplayRightArrow Think about \textbf{how} these computations can be parallelized! \\
    \DisplayRightArrow Think about \textbf{what} has to be communicated between different nodes!
\end{frame}
%%%%%%%%%%%%%%%%%%%%%%%%%%%%%%%%%%%%%%%%%%%%%%%%%%%%%%%%%%%%%%%%%%%%
