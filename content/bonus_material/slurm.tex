\section{SLURM}
%%%%%%%%%%%%%%%%%%%%%%%%%%%%%%%%%%%%%%%%%%%%%%%%%%%%%%%%%%%%%%%%%%%%
\begin{frame}{SLURM Workload Manager}
    The most important SLURM commands are:
    %\setbeamertemplate{description item}[align left]
    \begin{description}[ ]
      \item[\mintinline{text}{sinfo}]\hfill\\ \quad
      list available nodes including their operational state
      \item[\mintinline{text}{squeue}]\hfill\\\quad
      list running or pending jobs
      \item[\mintinline{text}{scancel JOB_ID}]\hfill\\\quad
      cancel to job identified by \texttt{JOB\_ID} and remove it from the queue
      \item[\mintinline{text}{srun --pty bash}]\hfill\\\quad
      start an interactive session on one node
      \item[\mintinline{text}{srun -N X ./simulate [OPTIONS]}]\hfill\\\quad
      run your simulation on $X$ compute nodes in parallel
      \item[\mintinline{text}{sbatch -N X -n X --wrap "./simulate [OPTIONS]"}]\hfill\\\quad
      queue your simulation on $X$ compute nodes in parallel for future execution
    \end{description}
    For more information, e.g., using \mintinline{text}{sbatch} with a script instead of \mintinline{text}{--wrap}, see the official \link{https://slurm.schedmd.com/}{SLURM homepage}.
\end{frame}
%%%%%%%%%%%%%%%%%%%%%%%%%%%%%%%%%%%%%%%%%%%%%%%%%%%%%%%%%%%%%%%%%%%%